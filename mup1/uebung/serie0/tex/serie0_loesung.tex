%%% Autor: 	Max Beining
%%% VORLAGE 

%% ALLGEMEINE EINSTELLUNGEN -------------------------------
\documentclass[11pt, a4paper, DIV=12]{scrartcl}
\usepackage[T1]{fontenc}
\usepackage[utf8]{inputenc}
\usepackage[ngerman]{babel}
%----------------------------------------------------------
%% ZUSATZPAKETE					% Verwendung (Bsp.)
\usepackage{caption}			% ...
\usepackage{enumitem}			% Aufzaehlung Abstand 0pt
\usepackage{nicefrac}


%% TABELLEN- UND ABBILDUNGEN NAME -------------------------
\captionsetup[figure]{justification=centerlast}
\captionsetup[table]{justification=centerlast}
\KOMAoption{captions}{tableheading}
\renewcaptionname{ngerman}{\figurename}{Abb.}
\renewcaptionname{ngerman}{\tablename}{Tab.}
\addtokomafont{caption}{\small}
\setcapindent{0pt}
%----------------------------------------------------------
\usepackage{listings}			% Anzeigen von Programm-Code
\usepackage{xcolor}				% Farberweiterung für PC

\definecolor{pblue}{rgb}{0.13,0.13,1}
\definecolor{pgreen}{rgb}{0,0.5,0}
\definecolor{pred}{rgb}{0.9,0,0}
\definecolor{pgrey}{rgb}{0.46,0.45,0.48}
\lstset{
	language=Java,
  	showspaces=false,
  	showtabs=false,
  	tabsize=4,
  	breaklines=true,
  	showstringspaces=false,
  	breakatwhitespace=true,
  	commentstyle=\color{pgreen},
  	keywordstyle=\color{pblue},
  	stringstyle=\color{pred},
  	basicstyle=\ttfamily,
  	moredelim=[il][\textcolor{pgrey}]
}


% keine Serifenlose Schrift in Gliederung
\setkomafont{disposition}{\normalcolor\bfseries}

%==============================================================================
%------------------------------BEGINN DES HAUPTEILS ---------------------------
%==============================================================================
\begin{document}%%%%%%%%%%%%%%%%%%%%%%%%%%%%%%%%%%%%%%%%%%%%%%%%%%%%%%%%%%%%%%%

\noindent
\textbf{Universität Leipzig, WS~20/21} \vspace*{-0.25cm}
\section*{Modellierung und Programmierung 1}
\subsection*{Zusatzaufgabe}
\noindent
Autor: Max Beining \\
\rule{\linewidth}{1pt} \\

%--------------------------------------

\noindent
\textbf{Aufgabenstellung}: \par
\noindent
Erste Schritte in einer Java-Entwicklungsumgebung durch das Einfügen von Code und die Ausgabe in der Konsole, Abgabe in Moodle kennenlernen \par

\vspace{0.8cm}
\noindent
\textbf{Programmcode}:
\begin{small}
\begin{lstlisting}[language=Java]
public class Name {

	public static void main(String[] args) {
		String name = "Max Beining";
		String eman = "";
		// Create Loop with condition i ist not equal to length "name"
		for (int i = 0; i != name.length(); i++) {
			eman += name.charAt(name.length() - i - 1);
		}
		// Print Statement
		System.out.println(eman);
	}
	
}
\end{lstlisting}
\end{small}

\vspace{0.8cm}
\noindent
\textbf{Ausgabe (Konsole)}:
\begin{small}
\begin{lstlisting}
	gninieB xaM
\end{lstlisting}
\end{small}









%==============================================================================
%------------------------------ENDE DES HAUPTEILS -----------------------------
%==============================================================================
\end{document}%%%%%%%%%%%%%%%%%%%%%%%%%%%%%%%%%%%%%%%%%%%%%%%%%%%%%%%%%%%%%%%%%
